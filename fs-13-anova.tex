% fs-13-anova.tex

\documentclass[xcolor=dvipsnames]{beamer}
\usepackage{teachbeamer}

\title{ANOVA}
\subtitle{{\CourseNumber}, BCIT}

\author{\CourseName}

\date{May 7, 2018}

% \begin{figure}[h]
% \includegraphics[scale=.3]{./diagrams/extrema1.png}
% \end{figure}

% Command             10pt    11pt    12pt
% \tiny               5       6       6
% \scriptsize         7       8       8
% \footnotesize       8       9       10
% \small              9       10      10.95
% \normalsize         10      10.95   12

\begin{document}

\begin{frame}
  \titlepage
\end{frame}

\begin{frame}
  \frametitle{Analysis of Variance}
One-way analysis of variance (ANOVA) is a method of testing the
equality of three or more population means by analyzing sample
variances. One-way analysis of variance is used with data categorized
with one factor (or treatment), so there is one characteristic used to
separate the sample data into the different categories.

\bigskip

In order to conduct one-way ANOVA, the requirements are (informally)
that
\begin{itemize}
\item the samples are independent
\item the populations are normally distributed, and
\item the population variances are approximately equal.
\end{itemize}
\end{frame}

\begin{frame}[fragile]
  \frametitle{Analysis of Variance}
  Consider the following R Statistics code. It corresponds to the
  narrative in Triola, page 562. The result is a test statistic
  following the $F$-distribution and a $p$-value that can be compared
  to the significance level. This type of ANOVA is always a right-hand
  one-tailed test.
% http://pages.stat.wisc.edu/~yandell/st571/R/anova.pdf
\begin{scriptsize}
\begin{alltt}
l<-c(85,90,107,85,100,97,101,64,111,100,76,136,100,90,135,104,149,99,107,
  99,113,104,101,111,118,99,122,87,118,113,128,121,111,104,51,100,113,
  82,146,107,83,108,93,114,113,94,106,92,79,129,114,99,110,90,85,94,127,
  101,99,113,80,115,85,112,112,92,97,97,91,105,84,95,108,118,86,89,100)
m<-c(78,97,107,80,90,83,101,121,108,100,110,111,97,51,94,80,101,92,100,
  77,108,85)
h<-c(93,100,97,79,97,71,111,99,85,99,97,111,104,93,90,107,108,78,95,78,
  86)
n<-c(length(l),length(m),length(h))
group<-rep(1:3,n)
y<-c(l,m,h)
data<-data.frame(y=y,group=factor(group))
fit<-lm(y~group,data)
anova(fit)
\end{alltt}
\end{scriptsize}
\end{frame}

\begin{frame}
  \frametitle{Anova: Lead and Intelligence Quotients (low)}
  \begin{figure}[h]
    \includegraphics[scale=0.35]{./diagrams/lead_l.png}
  \end{figure}
\end{frame}

\begin{frame}
  \frametitle{Anova: Lead and Intelligence Quotients (medium)}
  \begin{figure}[h]
    \includegraphics[scale=0.35]{./diagrams/lead_m.png}
  \end{figure}
\end{frame}

\begin{frame}
  \frametitle{Anova: Lead and Intelligence Quotients (high)}
  \begin{figure}[h]
    \includegraphics[scale=0.35]{./diagrams/lead_h.png}
  \end{figure}
\end{frame}

\begin{frame}
  \frametitle{Anova: Lead and Intelligence Quotients (boxplots)}
\begin{figure}[h]
  \includegraphics[scale=0.35]{./diagrams/lead_bp.png}
\end{figure}
\end{frame}

\begin{frame}[fragile]
  \frametitle{Anova: Lead and Intelligence Quotients}
  R Statistics yields the following output:
\begin{verbatim}
Analysis of Variance Table

Response: y
           Df  Sum Sq Mean Sq F value Pr(>F)  
group       2  1920.9  960.45  3.8646 0.0237 *
Residuals 117 29077.1  248.52                 
---
Signif. codes:  0 ‘***' 0.001 ‘**' 0.01 ‘*' 0.05 ‘.' 0.1 ‘ ' 1
\end{verbatim}
What is important to us is the test statistic, whose distribution is
the $F$-distribution, $3.8646$; and the $p$-value $0.0237$. Since
ANOVA is always one-tailed, area to the right, all you need to do is
compare the $p$-value to the significance level. ``If $p$ is low, the
NULL must go.''
\end{frame}

\begin{frame}
  \frametitle{$F$-Distribution}
  The $F$-distribution depends on two different degrees of freedom,
  which makes using a table of critical values awkward. We will use
  $p$-values provided by technology instead. A
  table with critical values is here: {\footnotesize
    \texttt{http://www.itl.nist.gov/div898/handbook/eda/section3/eda3673.htm}}.
  The shape of the $F$-distribution is similar to the shape of the
  $\chi^{2}$-distribution.
\begin{figure}[h]
  \includegraphics[scale=0.25]{./diagrams/fdist.png}
\end{figure}
\end{frame}

\begin{frame}
  \frametitle{Degrees of Freedom for the $F$-Distribution}
% nice explanation of one-way anova: https://people.richland.edu/james/lecture/m113/anova.html
  A variable that follows an $F$-distribution is the ratio of two
  independent chi-square variables divided by their respective degrees
  of freedom. Therefore, the $F$-distribution has two different
  degrees of freedom, one for the numerator and one for the
  denominator.
  \begin{itemize}
  \item Are all of the data values within any one group the same? No!
    So there is some within group variation. The within group is
    sometimes called the error group.
  \item Are all the sample means between the groups the same? No! So
    there is some between group variation. The between group is
    sometimes called the treatment group.
  \end{itemize}
\end{frame}

\begin{frame}
  \frametitle{Degrees of Freedom for the $F$-Distribution}
% nice explanation of one-way anova: https://people.richland.edu/james/lecture/m113/anova.html
  There are two sources of variation here. The between group and the
  within group. Let there be $k$ groups and $N$ data points within
  those $k$ groups.
  \begin{itemize}
  \item The between group degree of freedom (for the numerator) is
    $k-1$, just as it was for a categorical goodness-of-fit test.
  \item The within group degree of freedom (for the denominator) is
    $N-k$, just as it was for linear regression with $k=2$.
  \end{itemize}
  In the lead example, there are three groups (low, medium, high) with
  77, 22, and 21 individual data points respectively. The first degree
  of freedom is $3-1=2$; the second degree of freedom is $120-3=117$.
  A $0.05$ significance test would result in a critical value of
  \texttt{qf(0.95,2,117)} which equals $3.073763$ (this critical value
  is also displayed in Excel). We know from the ANOVA performed in
  Excel and in R Statistics that the test statistic is $3.8646$. We
  reject the null hypothesis.
\end{frame}

\begin{frame}
  \frametitle{One-Way ANOVA in Microsoft Excel}
For instructions, see {\tiny \texttt{http://www.excel-easy.com/examples/anova.html}}.
  \begin{figure}[h]
    \includegraphics[scale=0.5]{./diagrams/lead-with-anova.png}
  \end{figure}
\end{frame}

\begin{frame}[fragile]
  \frametitle{ANOVA Exercise} {\ubung} Copy and past the comma-separated value data (in cm) on the next slide. You have eight graduate students. They each measure the height of kindergarten children (six years of age, when there is no significant height difference between girls and boys). Can you rely on their measuring techniques to be consistent?
\end{frame}

\begin{frame}[fragile]
  \frametitle{ANOVA Exercise}
  \begin{scriptsize}
\begin{verbatim}
a<-c(119.8,116.5,120.3,111.2,107.6,116.1,110.0,114.9,118.7,121.7,115.6,109.4,
107.8,113.9,114.7,118.9,124.4,109.1,111.9,117.6,111.1,121.8,116.7,117.9,117.9)
b<-c(114.0,112.4,117.0,117.0,116.2,121.1,118.6,119.7,118.4,122.5,122.0,117.8,
124.2,112.8,116.4,112.2,113.7,114.7,116.8,122.4,111.7,115.9,114.4,109.6,123.2,
110.5,109.6,126.4)
c<-c(112.0,109.3,125.1,101.9,116.0,113.6,117.4,122.6,114.0,118.0,111.3,118.4,
117.0,121.3,118.6,119.8,120.0,121.5,122.3,120.7,123.2,106.3,121.8,123.2)
d<-c(116.3,110.5,109.1,117.5,124.6,108.1,117.9,111.7,126.9,110.9,115.7,110.8,
112.1,114.7,123.1,119.3,109.4,112.3,112.3,120.2,119.3,104.7,113.6,112.4,115.2,
112.9,121.8,128.8,115.7,114.6)
e<-c(123.1,119.2,113.3,114.2,110.3,120.6,111.4,119.7,106.7,112.1,113.3,119.3,
119.5,125.6,120.3,110.9,112.3,118.1,115.7,112.6,113.3,115.4,121.8,116.7,109.1,
117.8,113.0,107.4,117.3)
f<-c(114.1,116.5,114.5,108.4,111.6,118.1,116.6,114.0,116.3,109.6,117.0,112.0,
116.7,121.6,119.8,114.3,118.5,121.4,110.9,110.6,118.5,109.8,123.5,119.9,114.1,
115.2,123.8,113.6,110.3,111.7,106.6)
g<-c(111.9,116.4,111.8,105.8,111.9,117.1,113.8,111.9,107.5,111.7,116.3,118.4,
116.0,117.3,126.3,114.8,113.1,120.6,105.6,119.6,113.6,116.9,120.7,121.5,117.1,
117.4)
h<-c(114.4,120.0,115.1,121.6,123.5,120.3,107.8,110.7,112.6,115.6,110.2,116.3,
121.1,112.4,123.6,116.1,116.2,117.3,116.3,113.1,116.0,111.4,111.7,117.1,116.2,
113.9)
\end{verbatim}
  \end{scriptsize}
\end{frame}

\begin{frame}[fragile]
  \frametitle{R Statistics Code}
\begin{scriptsize}
\begin{alltt}
n<-c(length(a),length(b),length(c),length(d),
length(e),length(f),length(g),length(h))
group<-rep(1:8,n)
y<-c(a,b,c,d,e,f,g,h)
data<-data.frame(y=y,group=factor(group))
fit<-lm(y~group,data)
anova(fit)
\end{alltt}
\end{scriptsize}
The result is
\begin{verbatim}
Analysis of Variance Table

Response: y
           Df Sum Sq Mean Sq F value Pr(>F)
group       7  117.2  16.747  0.7035 0.6691
Residuals 211 5023.4  23.807 
\end{verbatim}
\end{frame}

\begin{frame}[fragile]
  \frametitle{ANOVA Exercise}
  {\ubung} Susan predicts that students will learn most effectively
  with a constant background sound, as opposed to an unpredictable
  sound or no sound at all. She randomly divides twenty-four students
  into three groups of eight. All students study a passage of text for
  30 minutes. Those in group 1 study with background sound at a
  constant volume in the background. Those in group 2 study with noise
  that changes volume periodically. Those in group 3 study with no
  sound at all. After studying, all students take a 10 point multiple
  choice test over the material. Test the appropriate null hypothesis
  using one-way ANOVA at a 0.05 significance level.

\begin{footnotesize}
\begin{verbatim}
+----------------+---+---+---+---+---+---+---+---+
| constant sound | 7 | 4 | 6 | 8 | 6 | 6 | 2 | 9 |
+----------------+---+---+---+---+---+---+---+---+
| random sound   | 5 | 5 | 3 | 4 | 4 | 7 | 2 | 2 |
+----------------+---+---+---+---+---+---+---+---+
| no sound       | 2 | 4 | 7 | 1 | 2 | 1 | 5 | 5 |
+----------------+---+---+---+---+---+---+---+---+
\end{verbatim}
\end{footnotesize}
\end{frame}

\begin{frame}
  \frametitle{Final Exam Practice}
  {\ubung} At a gas station, 40\% of customers fill their tanks. Of
  those who fill their tanks, 80\% pay with a credit card.
  \begin{enumerate}
  \item What is the probability that a customer fills their tank and
    pays with a credit card?
  \item What is the probability that either three or four out of ten
    customers fill their tank and pay by credit card?
  \item What is the probability that more than half of eight customers
    fill their tank and pay by credit card?
  \end{enumerate}
\end{frame}

\begin{frame}
  \frametitle{Final Exam Practice}
  {\ubung} At a certain time in the afternoon, London Heathrow sees on average 2
  planes landing per minute.
  \begin{enumerate}
  \item What is the probability of four or more planes landing in one
    minute?
  \item What is the probability that no plane will land in a
    particular minute?
  \end{enumerate}
\end{frame}

\begin{frame}
  \frametitle{Final Exam Practice}
  {\ubung} A classic story involves four carpooling students who
  missed a test and gave as an excuse a flat tire. On the makeup test,
  the instructor asked the students to identify the particular tire
  that went flat. If they really didn't have a flat tire, would they
  be able to identify the same tire? The author asked 41 other
  students to identify the tire they would select. The results are
  listed in the following table (except for one student who selected
  the spare). Use a 0.05 significance level to test the author's claim
  that the results fit a uniform distribution. What does the result
  suggest about the ability of the four students to select the same
  tire when they really didn't have a flat?

  \begin{tabular}{|l|r|}\hline
        left front  & 11 \\ \hline
        right front & 15 \\ \hline
        left rear   & 8  \\ \hline
        right rear  & 6  \\ \hline
  \end{tabular}
% Tire	Lett Front	Right Front	Left Rear	Right Rear
% Number Selected	11	15	8	6
\end{frame}

\begin{frame}
  \frametitle{Final Exam Practice}
  {\ubung} The police department in Madison, Connecticut, released the
  following numbers of calls for the different days of the week during
  a recent February that had 28 days: Monday (114); Tuesday (152);
  Wednesday (160); Thursday (164); Friday (179); Saturday (196);
  Sunday (130). Use a 0.01 significance level to test the claim that
  the different days of the week have the same frequencies of police
  calls. Is there anything notable about the observed frequencies?
\end{frame}

\begin{frame}[fragile]
  \frametitle{Final Exam Practice}
% Triola, page 515
  {\ubung} Listed below are numbers of enrolled students (in
  thousands) and numbers of burglaries for randomly selected large
  colleges in a recent year (based on data from the New York Times).
  Find the best predicted number of burglaries for Ohio State, which
  had an enrollment of 51,800 students. Does a 95\% confidence
  interval for the predicted value contain 329, which was the actual
  number of burglaries?
\begin{verbatim}
enrolment<-c(32,31,53,28,27,36,42,30,34,46)
burglaries<-c(103,103,86,57,32,131,157,20,27,161)
\end{verbatim}
\end{frame}

\begin{frame}
  \frametitle{End of Lesson}
Next Lesson: End of Term! Have a Happy Holiday!
\end{frame}

\end{document}

